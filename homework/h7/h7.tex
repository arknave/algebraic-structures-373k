\documentclass[12pt]{article}
\usepackage{amsfonts, amsmath, amsthm}

% a matter of taste
\setlength{\parskip}{1ex}
\setlength{\parindent}{0pt}

\newtheorem*{exer}{Exercise}

% A few simple macros for group theory.
\newcommand{\img}{\text{img }}
\newcommand{\lcm}{\text{lcm }}
\newcommand{\aut}{\text{Aut }}
\newcommand{\cycle}[1]{(\mathbf{#1})}

\begin{document}

\textbf{Homework 4 -- Algebraic Structures} \\

\hrule

% Problem list
\begin{minipage}{.80\linewidth}
    \flushleft
    Ch 11: 1.1, 1.2, 1.3, 1.8, 2.1, 3.1, 3.2, 3.3(a, d), 3.5, 3.6, 3.8,
    3.12, 3.13 \\
    % ``Pre-lecture problems''
    Pre-lect:  \\
\end{minipage}
\begin{minipage}{.20\linewidth}
    \flushright
    % whoami
    Blake Griffith
\end{minipage}

% % % % % % % % % % % % % % % % % % % % % % % % % % % % % % % % % % % % 
% Problems
% % % % % % % % % % % % % % % % % % % % % % % % % % % % % % % % % % % % 

\begin{exer}[11.1.1]

    Prove that $7 + 2^{1/3}$ and $\sqrt{3} + \sqrt{-5}$ are algebraic
    numbers.

\end{exer}

\begin{proof}
\end{proof}

% % % % % % % % % % % % % % % % % % % % % % % % % % % % % % % % % % % % 

\begin{exer}[11.1.2]

    Prove that, for $n \neq 0$, $\cos{2\pi / n}$ is an algebraic number.

\end{exer}

\begin{proof}
\end{proof}

% % % % % % % % % % % % % % % % % % % % % % % % % % % % % % % % % % % % 

\begin{exer}[11.1.3]

    Let $\mathbb{Q}[\alpha, \beta]$ denote the smallest subring of
    $\mathbb{C}$ containing the rational numbers $\mathbb{Q}$ and the
    elements $\alpha = \sqrt{2}$ and $\beta = \sqrt{3}$. Let $\gamma =
    \aspha + \beta$. Is $\mathbb{Q}[\alpha, \beta] =
    \mathbb{Q}[\gamma]$? Is $\mathbb{Z}[\alpha, \beta] =
    \mathbb{Z}[\gamma]$?

\end{exer}

\begin{proof}
\end{proof}

% % % % % % % % % % % % % % % % % % % % % % % % % % % % % % % % % % % % 

\begin{exer}[11.1.8]

    Determine the units in

    \begin{enumerate}
        \item $\mathbb{Z}/12\mathbb{Z}$
        \item $\mathbb{Z}/8\mathbb{Z}$
        \item $\mathbb{Z}/n\mathbb{Z}$
    \end{enumerate}

\end{exer}

\begin{proof}
\end{proof}

% % % % % % % % % % % % % % % % % % % % % % % % % % % % % % % % % % % % 

\begin{exer}[11.2.1]

    For which positive integers $n$ does $x^2 + x + 1$ divide $x^4 +
    3x^3 + x^2 + 7x + 5$ in $[\mathbb{Z}/(n)][x]$?

\end{exer}

\begin{proof}
\end{proof}

% % % % % % % % % % % % % % % % % % % % % % % % % % % % % % % % % % % % 

\begin{exer}[11.3.1]

    Prove that an ideal of a ring $R$ is a subgroup of the additive
    group $R^+$.

\end{exer}

\begin{proof}
\end{proof}

% % % % % % % % % % % % % % % % % % % % % % % % % % % % % % % % % % % % 

\begin{exer}[11.3.2]

    Prove that every nonzero ideal in the ring of Gauss integers
    contains a nonzero integer.

\end{exer}

\begin{proof}
\end{proof}

% % % % % % % % % % % % % % % % % % % % % % % % % % % % % % % % % % % % 

\begin{exer}[11.3.3 a and d]

    \begin{enumerate}
        \item $\mathbb{R}[x, y] \rightarrow \mathbb{R}$ defined by $f(x,
            y) \leadsto f(0, 0)$

        \item $\mathbb{Z}[x] \rightarrow \mathbb{C}$ definede by $x
            \leadsto \sqrt{2} + \sqrt{3}$

    \end{enumerate}
\end{exer}

\begin{proof}
\end{proof}

% % % % % % % % % % % % % % % % % % % % % % % % % % % % % % % % % % % % 

\begin{exer}[11.3.5]

    The derivative of a polynomial $f$ with coefficients in a field $F$
    is defined by the calculus formula $(a_n x^n + \dots + a_1 x + a_0)'
    = n a_n x^{n-1} + \dots + 1 a_1$. The integre coefficients are
    interpreted in $F$ using the unique homomorphism $\mathbb{Z}
    \rightarrow F$.

    \begin{enumerate}
        \item Prove the product rule $(fg)' = f'g + fg'$ and the chain
            rule $(f \circ g)' = (f' \circ g)g'$.

        \item Let $\alpha$ be an element of $F$. Prove that $\alpha$ is
            a multiple root of a polynomial $f$ if and only if it is a
            common root of $f$ and of its derivative $f'$.

    \end{enumerate}
\end{exer}

\begin{proof}
\end{proof}

% % % % % % % % % % % % % % % % % % % % % % % % % % % % % % % % % % % % 

\begin{exer}[11.3.6]

    An \textit{automorphism} of a ring $R$ is an isomorphism from $R$ to
    itself. Let $R$ be a ring. and let $f(y)$ be a polynomial in one
    variable with coefficients in $R$. Prove that the map $R[x, y]
    \rightarrow R[x, y]$ defined by $x \leadsto x + f(y)$, $y \leadsto
    y$ is an automorphism of $R[x, y]$.

\end{exer}

\begin{proof}
\end{proof}

% % % % % % % % % % % % % % % % % % % % % % % % % % % % % % % % % % % % 

\begin{exer}[11.3.8]

    Let $R$ be a ring of prime characteristic $p$. Prove that the map $R
    \rightarrow R$ defined by $x \leadsto x^p$ is a ring homomorphism.
    (It is called the \textit{Frobbenius map}.)

\end{exer}

\begin{proof}
\end{proof}

% % % % % % % % % % % % % % % % % % % % % % % % % % % % % % % % % % % % 

\begin{exer}[11.3.12]

    Let $I$ and $J$ be ideals of a ring $R$. Prove that the set $I + J$
    of elements of the form $x + y$, with $x$ in $I$ and $y$ in $J$, is
    an ideal. This ideal is called the \textit{sum} of the ideals $I$
    and $J$. 

\end{exer}

\begin{proof}
\end{proof}

% % % % % % % % % % % % % % % % % % % % % % % % % % % % % % % % % % % % 

\begin{exer}[11.3.13]

    Let $I$ and $J$ be ideals of a ring $R$. Prove that the intersection
    $I \cap J$ is an ideal. Show by example that the set of products
    $\{xy | x \in I, y \in J\}$ need not be an ideal, but that the set
    of finite sums $\Sigma x_\nu y_\nu$ of products of elements of $I$
    and $J$ is an ideal. This ideal is called the \textit{product
    ideal}, and is denoted by $IJ$. Is there a relation between $IJ$ and
    $I\capJ$?

\end{exer}

\begin{proof}
\end{proof}

% % % % % % % % % % % % % % % % % % % % % % % % % % % % % % % % % % % % 
% Pre-lecture Problems
% % % % % % % % % % % % % % % % % % % % % % % % % % % % % % % % % % % % 
\section*{Pre-Lecture Problems}

\begin{exer}[7.5.1]

    A question you should be able to answer before class.

\end{exer}

\begin{proof}

    \dots

\end{proof}

% % % % % % % % % % % % % % % % % % % % % % % % % % % % % % % % % % % % 

\end{document}
