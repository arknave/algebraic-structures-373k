\documentclass[12pt]{article}
\usepackage{amsfonts, amsmath, amsthm}

% a matter of taste
\setlength{\parskip}{1ex}
\setlength{\parindent}{0pt}

\newtheorem*{exer}{Exercise}

% A few simple macros for group theory.
\newcommand{\img}{\text{img }}
\newcommand{\lcm}{\text{lcm }}
\newcommand{\aut}{\text{Aut }}
\newcommand{\cycle}[1]{(\mathbf{#1})}

\begin{document}

\textbf{Homework 6 -- Algebraic Structures} \\

\hrule

% Problem list
\begin{minipage}{.80\linewidth}
    \flushleft
    Ch 7: 7.2, 7.4, 7.5, 7.9, 7.10, 8.3, 8.4, 8.5, M.5, M.6 \\ 
    % ``Pre-lecture problems''
    Pre-lect: 5.1 \\
\end{minipage}
\begin{minipage}{.20\linewidth}
    \flushright
    % whoami
    Blake Griffith
\end{minipage}

% % % % % % % % % % % % % % % % % % % % % % % % % % % % % % % % % % % % 
% Problems
% % % % % % % % % % % % % % % % % % % % % % % % % % % % % % % % % % % % 

\begin{exer}[7.7.2]

    Let $G_1 \subset G_2$ be groups whose orders are divisible by $p$,
    and let $H_1$ be a Sylow p-subgroup of $G_1$. Prove that there is a
    Sylow p-subgroup $H_2$ of $G_2$ such that $H_1 = H_2 \cap G_1$.

\end{exer}

\begin{proof}

    Since $H_1$ is a subgroup of $G_2$ and a p-group, by the second
    sylow theorem it must be contained in a sylow p-subgroup of $G_2$.

    Let $H_2$ be this subgroup. Note that this subgroup must exist by
    the first sylow theorem.

    Then $H_1$ is contained in $H_2$ and $G_1$, so it must be contained
    in $H_2 \cap G_1$. 

    Now we must show $H_1 \supset H_2 \cap G_1$. Let $h$ be some element
    in $H_2 \cap G_1$. Since $h$ is an element of the p-group $H_2$, and
    in $G_2$ it must generate a p-group in $G_{2}$. Since $h$ is an
    element of a p-group in $G_2$ it must be contained in a sylow
    p-group of $G_1$, we choose this to be $H_1$. So $H_1 \subset H_2
    \cap G_1$. Since there is containment both ways we have $H_1 = H_2
    \cap G_1$. 

\end{proof}

% % % % % % % % % % % % % % % % % % % % % % % % % % % % % % % % % % % % 

\begin{exer}[7.7.4]

    \begin{enumerate}
        \item Prove that no simple group has order $pq$, where $p$ and
            $q$ are prime.

        \item Prove that no simple group has order $p^2q$, where $p$ and
            $q$ are prime.

    \end{enumerate}

\end{exer}

\begin{proof}

    \begin{enumerate}
        \item Without loss of generality suppose that $p > q$. Then
            there are $s$ sylow p-subpgroups where $s$ must divide $q$.
            So $s$ is either $q$ or $1$. We also know $s$ must be
            congruent to $p \mod 1$. So $s = kp + 1$. But since $p > q$
            there is not $k$ that satisfies $s = q$. So $s$ must be $1$. 

            Since there is only one sylow p-subgroup, it has no other
            conjugate subgroups, by the second sylow theorem. Therefore
            the p-subgroup must be normal. Therefore the group is not
            simple.

        \item Consider the case where $p > q$. Then let $s$ be the
            number of p-subgroups. $s$ must divide $q$, so it is $1$ or
            $q$. And $s$ must satisfy $s =
            pk + 1$ for some integer $k$. Since $p > q$ this is only
            satified when $s = 1$. Making the sylow p-subgroup normal
            and the group non-simple.

            For the case where $p < q$ we let the number of sylow
            q-subgroups be $n_q$. Then applying the sylow theorems we
            require $n_q = 1, p, p^2 = kq + 1$. $n_q$ cannot be $p$
            since $p \neq kq + 1$ for any $k$. 

            Now suppose $n_q = p^2$, then since the sylow q-subgroups
            are prime order, they intersect trivially, otherwise they
            would be the same group. So $p^2$ sylow q-subgroups account
            for $p^2(q - 1)$ non-identity elements. 

            Now if we consider the number of sylow p-subgroups $n_p$.
            There can be 1 or $q$ of these. And they must share no
            non-identity elements with the sylow q-subgroups. Because
            those elements would generate q order subgroups. If there
            are $q$ sylow p-subgroups there must be more than $p^2$
            elements in these groups or they would all be the same. But
            if there are more than $p^2$ unique elements in the sylow
            p-subgroups, then adding these to the number of elements in
            the sylow q-subgroups would give more than $p^2q$ elements.
            
            So either there is one sylow p-subgroup or one sylow
            q-subgroup. In either case there is a normal subgroup in the
            group, so it is not simple.

            For the case $p = q$, we have a group of order $p^3$. Recall
            that p groups have a non-trivial center and that the center
            is normal in the group. Therefore a group of order $p^3$
            cannot be simple.

    \end{enumerate}

\end{proof}

% % % % % % % % % % % % % % % % % % % % % % % % % % % % % % % % % % % % 

\begin{exer}[7.7.5]

    Find Sylow 2-subgroups of $D_{10}$.

\end{exer}

\begin{proof}

    Recall that the $D_{10}$ is order $20$, and we choose the
    representation $r$ a rotation of $36^\circ$ and $l$ is a reflection
    across the vertical axis of symmetry, so $r^{10} = l^2 = 1$. We also
    have Since the groups order is $20 = 2^2 5$ the group has either $5$
    or $1$ sylow 2-subgroups. And $r^5 l = l r^5$. 

    Notice that there is at least one obvious subgroup of order 4 $\{1,
    l, r^5, lr^5\}$. We can't elimiante the chance of more sylow
    4-subgroups so we look for more. These would contain elemnts of
    order 2, so we keep $r^5$ and look for more. We find that $\{1, r^5,
    rl, r^6l\}$ is a sylow p-subgroup. So there must be at least 3 more
    sylow 2-subgroups. 

    Notice that $r^5 \times r^xl = r^y l \rightarrow r^5 l =
    r^{y-x}l \rightarrow y - x = 5$ is a requirement if we have two
    elements of the form $r^xl$ and $r^y l$. So the remaining groups
    are:
    \begin{align*}
        \{1, r^5, r^2l, r^7l\} \\
        \{1, r^5, r^3l, r^8l\} \\
        \{1, r^5, r^4l, r^9l\}
    \end{align*}

\end{proof}

% % % % % % % % % % % % % % % % % % % % % % % % % % % % % % % % % % % % 

\begin{exer}[7.7.9]

    Classify groups of order \textbf{(1)} 33 \textbf{(2)} 18.

\end{exer}

\begin{proof}

    \begin{enumerate}
        \item Note that $33 = 11 \times 3$. Let the number of sylow
            11-subgroups be $s$. Then by the third sylow theorem $s$
            must divide 3, and $s = k11 + 1$. The only choice of $s$
            that works here is $1$. So there is only 1 sylow
            11-subgroup. Now let the number of sylow 3-subgroups be $r$.
            Thene $r$ must divide 11 and $r = k3 + 1$. The only choice
            here is $r = 1$. So there is 1 sylow 3-subgroup.

            Any group of order 33 must contain bothe of the cyclic
            subgroups. The product of the sylow groups must be in the
            group. And since the product of the sylow groups is order
            33, it must be equal to the group. 

            So all groups of order 33 are isomorphic to $C_{3} C_{11}$

        \item Note that $18 = 3^2 \times 2$. We call the sylow
            3-subgroups $S_3$ and the sylow 2-subgroups $S_2$. We call
            the number of these sylow groups $N_3$ and $N_2$
            respectively.

            Applying the third sylow theorem shows $N_3 = 1$ and $S_3$
            is always normal. And $N_2 = 1, 3,$ or $9$. So we have
            several cases to address for $N_2$.

            $N_2 = 1 \implies S_2$ is normal. And since elements in
            $S_2$ are order 2, while those in $S_3$ can be 3 or 9, $S_2$
            and $S_3$ must intersect trivially so $G = S_3 \times S_2$.
            But there are several possibilities for $S_3$: $C_9$ and $C_3
            \times C_3$. 

            So the two of the isomorphisim classes are $C_3 \times C_3
            \times C_2$ and $C_9 \times C_2 = C_{19}$.

            $N_2 = 3$ then $|S_3$\dots


    \end{enumerate}

\end{proof}

% % % % % % % % % % % % % % % % % % % % % % % % % % % % % % % % % % % % 

\begin{exer}[7.7.10]

    Prove that the only simple groups of order $<$ 60 are the groups of
    prime order.

\end{exer}

\begin{proof}

    First we list all the numbers less than 60 that are not prime 

    \begin{verbatim}
     4   6   8 9 10    12    14 15 16    18    20 21 22    24 25
     26 27 28    30    32 33 34 35 36    38 39 40    42    44 45 46
     48 49 50 51 52    54 55 56 57 58
     \end{verbatim}

     Next we note that from previous problems that groups of order $pq$
     are not simple.

     \begin{verbatim}
             8         12          16    18    20          24   
        27 28    30    32          36          40    42    44      
           50    52    54    56      
     \end{verbatim}

     Next recall that groups of order $p^2q$ cannot be simple.
        
     \begin{verbatim}
                                   16                      24   
                 30    32          36          40    42            
                       54    56      
     \end{verbatim}

     Now recall that every group that is of order $p^n$ has a
     non-trivial center, and therefore a normal subgroup. Eliminating
     these gives:

     \begin{verbatim}
                                                           24   
                 30                36          40    42            
                       54    56      
     \end{verbatim}

     Now we consider these iduvidually. For $24 = 2^3 \times 3$ there can be
     3 or 1 sylow 2-subgroups and 1 or 4 sylow 3-subgroups. However
     There cannot be both 3 sylow 2-subgroups and 4 sylow 3-subgroups
     since the groups only intersect trivially and this account for 32
     elements. So at least one of the groups is normal.

     $30 = 5 \times 6 = 3 \times 10 = 5 \times 6$. So there can be 1 or
     10 sylow 3-subgroups and 1 or 6 sylow 5-subgroups. However there
     are to many elements if there are both 10 sylow 3-subgroups and 6
     sylow 5-subgroups, so there must be only one of either.

     $36 = 2^2 \times 3^2$ So there can be either 1, 3, or 9 sylow
     2-subgroups and 1 or 4 sylow 3-subgroups. But counting elements
     shows there cannot be both 4 sylow 3-subgroups and 3 or 9 sylow
     2-subgroups. So there is only 1 of either sylow 3-subgroups or
     sylow 2-subgroups.

     $40 = 2^3 \times 5$, we require there be either 1, 2, 4, or 8 sylow
     5-subgroups and that the number of sylow 5-subgroups is congruent
     to 1 mod 5. This is only the case for 1.

     $42 = 7 \times 3 \times 2$, we require there be either 1, 2, 3, or
     6 sylow 7-subgroups and that the number of sylow 7-subgroups be
     congruent to 1 mod 7. This is only the case for 1.

     $54 = 3^3 \times 2$, we require there be either 1 or 2 sylow
     3-subgroups and that the number of sylow 3-subgroups be congruent
     to 1 mod 2. This is only the case for 1.

     $56 = 2^3 \times 7$, so there can be either 1 or 8 sylow
     7-subgroups and 1 or 7 sylow 2 subgroups. But counting elements
     shows there cannot be both 8 sylow 7-subgroups and 7 sylow
     2-subgroups. So there is only 1 of either.

\end{proof}

% % % % % % % % % % % % % % % % % % % % % % % % % % % % % % % % % % % % 

\begin{exer}[7.8.3]

    Determine the class equations of the groups of order 12.

\end{exer}

\begin{proof}

    We are given the 5 isomorphism classes of groups of order 12. The
    first two of these are $C_4 \times C_3$ and $C_2
    \times C_2 \times C_3$. Since these are abellian their class
    equaiton is 
    \[
        1 \times 12
    \]

    The next isomorphism class is the alternating group $A_4$, for which
    we are given that the sylow 3-subgroup, $K$, is not normal, so there
    are four sylow 3-subgroups.

    So by the second sylow theorem each sylow 3-subgroup is conjugate to
    the others. We also know these groups intersect trivially since they
    are of prime order. Therefore each element in a sylow 3-subgroup is
    conjugate to an element in the other sylow 3-subgroups. So each
    non-identity element in the sylow 3-subgroups has a conjugacy class
    of order 4. This gives $ + 4 + 4$.

    We are also given that $H$ must be normal. So conjugating $H$ with
    any element gives us back $H$. So every non identity element in $H$ must be in the
    same conjugacy class, which gives $+ 3$. So with the identity we
    have:
    \[
        1 + 4 + 4 + 3
    \]

    The next isomorphism class is the dihedral group $D_6$. Here $K$,
    the sylow 3-subgroup is normal and $H$ the sylow 2-subgroup is not
    normal. 

    Since $H$ is not normal there must be 3 sylow 2-subgroups, which are
    all conjugate subgroups to each other. So each non-identity element
    in the sylow 2-subgroup has 3 elements in its conjugacy class. This
    gives $+ 3 + 3 + 3$.

    Since $K$ is normal, all of its non-identity elements are in the
    same conjugacy class, which gives $+2$.


    

\end{proof}

% % % % % % % % % % % % % % % % % % % % % % % % % % % % % % % % % % % % 

\begin{exer}[7.8.4]

    Prove that a group of order $n = 2p$, where $p$ is a prime, is
    either cyclic or dihedral.

\end{exer}

\begin{proof}

    Applying sylow's third theorem we see that there is 1 sylow
    p-subgroup $S_p$. And either 1 or p sylow 2-subgroups $S_2$. 

    Let $x \in S_p$ and $y \in S_2$. Then we have $x^p = x^2 = 1$, and
    $yxy = x^a$ for some $a \in (1 \dots p-1)$. So
    \[
        x = y^{-2}xy^2 = y^{-1}x^ay = y^{-1} a \dots a y =
        y^{-1}\underbrace{a y^{-1}y a \dots y^{-1}y a}_\text{$a$ times}y =
        x^{a^2}
    \]
    Or $x = x^{a^2}$. So $a^2$ is congruent to 1 mod p. So p divides
    $a^2 - 1 = (a + 1)(a - 1)$. So $a = p + 1 = 1$ or $a = p - 1 = -1$.

    For the case $a = 1$ we have $yxy = x$. So $S_2$ commutes with
    $S_p$, so we have the cyclic group $C_p \times C_2$. 

    Otherwise $yxy = x^{-1}$. This along with the other requirements is the
    definition of the dihedral group. 

\end{proof}

% % % % % % % % % % % % % % % % % % % % % % % % % % % % % % % % % % % % 

\begin{exer}[7.8.5]

    Let $G$ be a nonabelian group of order 28 whose sylow 2 subgroups
    are cyclic.

    \begin{enumerate}
        \item Determine the numbers of sylow 2 subgroups and sylow 7
            subgroups.

        \item Prove that there is at most one isomorphism class of such
            groups.

        \item Determine the numbers of elements of each order, and the
            class equation of $G$.
    \end{enumerate}

\end{exer}

\begin{proof}

    \begin{enumerate}
        \item We denote the sylow 7-subgroups and sylow 2-subgroups as
            $S_7$ and $S_2$ respectively. The number of these groups is
            $N_7$ and $N_2$. 

            We know by sylow's third theorem that $N_7$ must divide 1,
            2, or 4 and be congruent to 1 mod 7. So $N_7 = 1$. 

            Likewise we can have $N_2 = 1$ or $7$. But if $N_2 = 1$.
            $G = S_2 \times S_7$ which is abellian. So
            \[
                N_2 = 7 \quad N_7 = 1
            \]

        \item From $S_2$'s order we could deduce that it could be $C_4$
            or $K_4$. But if it were $C_4$, the seven $C_4$ would have
            21 unique elements. Leaving no room for products with $S_7$.
            So it must be $K_4$. As before, we don't want every $S_2$ to
            have 3 new unique elements. They can all share at most 2
            elements, 1 and what we call $a$. $a$'s inverse must be in
            every group so $a^2 = 1$. So each $S_2$ has a unique $b_i$
            and $c_i$ such that $b^{2}_{i} = c^{2}_{i} = 1$\dots

        \item We know there is one cyclic group of order 7 which
            contributes 6 elements of order 7.

    \end{enumerate}

\end{proof}

% % % % % % % % % % % % % % % % % % % % % % % % % % % % % % % % % % % % 

\begin{exer}[7.M.5]

    Let $H$ and $N$ be subgroups of a group $G$, and assume that $N$ is
    a normal subgroup.

    \begin{enumerate}
        \item Determine the kernels of the restrictions of the canonical
            homomorphism $\pi : G \rightarrow G/N$ ot the subgroups $H$
            and $HN$. 

        \item Applying First Isomorphism Theorem to these restrictions,
            prove the \textit{Second Isomorphism Theorem}: $H/(H \cap
            N)$ is isomorphic to $(HN)/N$. 
    \end{enumerate}

\end{exer}

\begin{proof}

\end{proof}

% % % % % % % % % % % % % % % % % % % % % % % % % % % % % % % % % % % % 

\begin{exer}[7.M.6]

    Let $H$ and $N$ be normal subgroups of a group $G$ such that $H
    \supset N$. Let $\overline{H} = H/N$ and $\overline{G} = G/N$.

    \begin{enumerate}
        \item Prove that $\overline{H}$ is a normal subgroup of
            $\overline{G}$. 
        \item Use the composed homomorphism $G \rightarrow \overline{G}
            \rightarrow \overline{G}/\overline{H}$ to prove the
            \textit{Third Isomorphism Theorem}: $G/H$ is isomorphic to
            $\overline{G} / \overline{H}$.
            
    \end{enumerate}

\end{exer}

\begin{proof}

\end{proof}

\end{document}
