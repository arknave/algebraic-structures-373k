\documentclass[12pt]{article}
\usepackage{amsfonts, amsmath, amsthm}

% a matter of taste
\setlength{\parskip}{1ex}
\setlength{\parindent}{0pt}

\newtheorem*{exer}{Exercise}

% A few simple macros for group theory.
\newcommand{\img}{\text{img }}
\newcommand{\lcm}{\text{lcm }}
\newcommand{\aut}{\text{Aut }}
\newcommand{\cycle}[1]{(\mathbf{#1})}

\begin{document}

\textbf{Homework 6 -- Algebraic Structures} \\

\hrule

% Problem list
\begin{minipage}{.80\linewidth}
    \flushleft
    Ch 7: 7.2, 7.4, 7.5, 7.9, 7.10, 8.3, 8.4, 8.5, M.5, M.6 \\ 
    % ``Pre-lecture problems''
    Pre-lect: 5.1 \\
\end{minipage}
\begin{minipage}{.20\linewidth}
    \flushright
    % whoami
    Blake Griffith
\end{minipage}

% % % % % % % % % % % % % % % % % % % % % % % % % % % % % % % % % % % % 
% Problems
% % % % % % % % % % % % % % % % % % % % % % % % % % % % % % % % % % % % 

\begin{exer}[7.7.2]

    Let $G_1 \subset G_2$ be groups whose orders are divisible by $p$,
    and let $H_1$ be a Sylow p-subgroup of $G_1$. Prove that there is a
    Sylow p-subgroup $H_2$ of $G_2$ such that $H_1 = H_2 \cap G_1$.

\end{exer}

\begin{proof}

    Put a tautology here.

\end{proof}

% % % % % % % % % % % % % % % % % % % % % % % % % % % % % % % % % % % % 

\begin{exer}[7.7.4]

    \begin{enumerate}
        \item Prove that no simple group has order $pq$, where $p$ and
            $q$ are prime.

        \item Prove that no simple group has order $p^2q$, where $p$ and
            $q$ are prime.

    \end{enumerate}

\end{exer}

\begin{proof}

\end{proof}

% % % % % % % % % % % % % % % % % % % % % % % % % % % % % % % % % % % % 

\begin{exer}[7.7.5]

    Find Sylow 2-subgroups of $D_{10}$.

\end{exer}

\begin{proof}

\end{proof}

% % % % % % % % % % % % % % % % % % % % % % % % % % % % % % % % % % % % 

\begin{exer}[7.7.9]

    Classify groups of order \textbf{(1)} 33 \textbf{(2)} 18.

\end{exer}

\begin{proof}

\end{proof}

% % % % % % % % % % % % % % % % % % % % % % % % % % % % % % % % % % % % 

\begin{exer}[7.7.10]

    Prove that the only simple groups of order $<$ 60 are the groups of
    prime order.

\end{exer}

\begin{proof}

\end{proof}

% % % % % % % % % % % % % % % % % % % % % % % % % % % % % % % % % % % % 

\begin{exer}[7.8.3]

    Determine the class equations of the groups of order 12.

\end{exer}

\begin{proof}

\end{proof}

% % % % % % % % % % % % % % % % % % % % % % % % % % % % % % % % % % % % 

\begin{exer}[7.8.4]

    Prove that a group of order $n = 2p$, where $p$ is a prime, is
    either cyclic or dihedral.

\end{exer}

\begin{proof}

\end{proof}

% % % % % % % % % % % % % % % % % % % % % % % % % % % % % % % % % % % % 

\begin{exer}[7.8.5]

    Let $G$ be a nonabelian group of order 28 whose sylow 2 subgroups
    are cyclic.

    \begin{enumerate}
        \item Determine the numbers of sylow 2 subgroups and sylow 7
            subgroups.

        \item Prove that there is at most one isomorphism class of such
            groups.

        \item Determine the numbers of elements of each order, and the
            class equation of $G$.
    \end{enumerate}

\end{exer}

\begin{proof}

\end{proof}

% % % % % % % % % % % % % % % % % % % % % % % % % % % % % % % % % % % % 

\begin{exer}[7.M.5]

    Let $H$ and $N$ be subgroups of a group $G$, and assume that $N$ is
    a normal subgroup.

    \begin{enumerate}
        \item Determine the kernels of the restrictions of the canonical
            homomorphism $\pi : G \rightarrow G/N$ ot the subgroups $H$
            and $HN$. 

        \item Applying First Isomorphism Theorem to these restrictions,
            prove the \textit{Second Isomorphism Theorem}: $H/(H \cap
            N)$ is isomorphic to $(HN)/N$. 
    \end{enumerate}

\end{exer}

\begin{proof}

\end{proof}

% % % % % % % % % % % % % % % % % % % % % % % % % % % % % % % % % % % % 

\begin{exer}[7.M.6]

    Let $H$ and $N$ be normal subgroups of a group $G$ such that $H
    \supset N$. Let $\overline{H} = H/N$ and $\overline{G} = G/N$.

    \begin{enumerate}
        \item Prove that $\overline{H}$ is a normal subgroup of
            $\overline{G}$. 
        \item Use the composed homomorphism $G \rightarrow \overline{G}
            \rightarrow \overline{G}/\overline{H}$ to prove the
            \textit{Third Isomorphism Theorem}: $G/H$ is isomorphic to
            $\overline{G} / \overline{H}$.
            
    \end{enumerate}

\end{exer}

\begin{proof}

\end{proof}

\end{document}
