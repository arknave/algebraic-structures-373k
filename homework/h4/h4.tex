\documentclass[12pt]{article}
\usepackage{amsfonts, amsmath, amsthm}

\setlength{\parskip}{1ex}
\setlength{\parindent}{0pt}

\newtheorem*{exer}{Exercise}

\newcommand{\img}{\text{img }}
\newcommand{\lcm}{\text{lcm }}
\newcommand{\cycle}[1]{(\mathbf{#1})}

\begin{document}

\textbf{Homework 4 -- Algebraic Structures} \\

\hrule

\begin{minipage}{.80\linewidth}
    \flushleft
    Ch 6: 7.10, 8.3, 8.4, M.2, M.3; Ch 7: 1.2, 2.3, 2.4, 2.7, 2.8, 2.9,
    2.13, 2.14, 2.17 \\
    Pre-lect: Ch 6: 7.1, 8.2; Ch 7: 1.1, 2.2 \\
\end{minipage}
\begin{minipage}{.20\linewidth}
    \flushright
    Blake Griffith
\end{minipage}

\begin{exer}[6.7.10]
    \begin{enumerate}
        \item Describe the orbit and the stabilizer of the matrix
            $\left[
                \begin{array}{cc}
                    1 & 0 \\
                    0 & 2 
                \end{array}
            \right]$
            under conjugation in the general linear group $GL_N (
            \mathbb{R})$.

        \item Interpreting the matrix in $GL_2(\mathbb{F}_5)$, find
            the order of the orbit.
    \end{enumerate}
\end{exer}

\begin{proof}

\end{proof}

% % % % % % % % % % % % % % % % % % % % % % %

\begin{exer}[6.8.3]
    Exhibit the bijective map (6.8.4) explicitly, when $G$ is the
    dihedral group $D_4$ and $S$ is the set of vertices of a square.
\end{exer}

\begin{proof}

\end{proof}

% % % % % % % % % % % % % % % % % % % % % % %

\begin{exer}[6.8.4]
    Let $H$ be the stabilizer of the index $1$ for the operation of the
    symmetric group $G = S_n$ on the set of indices $\{1, \dots, n\}$.
    Describe the left cosets of $H$ in $G$ and the map (6.8.4) in this
    case.
\end{exer}

\begin{proof}

\end{proof}

% % % % % % % % % % % % % % % % % % % % % % %

\begin{exer}[6.M.2]
    \begin{enumerate}
        \item Prove that the set Aut $G$ of automorphisms of a group $G$
            forms a group, the law of composition being the composition
            of functions.
        \item Prove that the map $\phi : G \rightarrow \text{ Aut} G$
            defined by $g \leadsto$ (conjugation by $g$) is a
            homomorphism, and determine its kernel.
        \item The automorphisms that are obtained as conjugation by a
            group element are called inner automorphisms. Prove that the
            set of inner automorphisms, the image of $\phi$, is a normal
            subgroup of the group Aut $G$.
    \end{enumerate}
\end{exer}

\begin{proof}

\end{proof}

% % % % % % % % % % % % % % % % % % % % % % %

\begin{exer}[6.M.3]
    Determine the groups of automorphisms (see Exercise M.2) of the
    group \textbf{(a)} $C_4$ \textbf{(b)} $C_6$, \textbf{(c)} $C_2
    \times C_2$, \textbf{(d)} $D_4$, \textbf{(e)} the quaternion group
    $H$.
\end{exer}

\begin{proof}

\end{proof}

% % % % % % % % % % % % % % % % % % % % % % %

\begin{exer}[7.1.2]
    Let $H$ be a subgroup of a group $G$. Describe the orbits for the
    operation of $H$ on $G$ by left multiplication.
\end{exer}

\begin{proof}

\end{proof}

% % % % % % % % % % % % % % % % % % % % % % %

\begin{exer}[7.2.3]
    A group $G$ of order 12 contains a conjugacy class of order 4. Prove
    that the center of $G$ is trivial.
\end{exer}

\begin{proof}

\end{proof}

% % % % % % % % % % % % % % % % % % % % % % %

\begin{exer}[7.2.4]
    Let $G$ be a group, and let $\phi$ be the \textbf{n}th power
    map:$\phi(x) = x^n$. What can be said about how $\phi$ acts on
    conjugacy classes?
\end{exer}

\begin{proof}

\end{proof}

% % % % % % % % % % % % % % % % % % % % % % %

\begin{exer}[7.2.7]
    Rule out as many as you can, as class equations for a group of order
    10:
    \[
        1 + 1 + 1 + 2 + 5, \quad 1 + 2 + 2 + 5, \quad 1 + 2 + 3 + 4,
        \quad 1 + 1 + 2 + 2 + 2 + 2
    \]
\end{exer}

\begin{proof}

\end{proof}

% % % % % % % % % % % % % % % % % % % % % % %

\begin{exer}[7.2.8]
    Determine the possible class equations of nonabelian groups of order
    \textbf{(a)}8, \textbf{(b)}21.
\end{exer}

\begin{proof}

\end{proof}

% % % % % % % % % % % % % % % % % % % % % % %

\begin{exer}[7.2.9]
    Determine the class equations for the following groups: \textbf{(a)}
    the quaternion group, \textbf{(b)} $D_4$, \textbf{(c)} $D_5$,
    \textbf{(d)} the subgroup of $GL_2(\mathbb{F}_3)$ of invertible
    upper triangular matrices.
\end{exer}

\begin{proof}

\end{proof}

% % % % % % % % % % % % % % % % % % % % % % %

\begin{exer}[7.2.13]
    Let $N$ be a normal subgroup of a group $G$. Suppose that $|N| = 5$
    and that $|G|$ is an odd integer. Prove that $N$ is contained in the
    center of $G$.
\end{exer}

\begin{proof}

\end{proof}

% % % % % % % % % % % % % % % % % % % % % % %

\begin{exer}[7.2.14]
    The class equation of a group $G$ is $ 1 + 4 + 5 + 5 + 5$.
    \begin{enumerate}
        \item Does $G$ have a subgroup of order 5? If so, is it a normal
            subgroup?
        \item Does $G$ have a subgroup of order 4? If so, is it a normal
            subgroup?
    \end{enumerate}
\end{exer}

\begin{proof}

\end{proof}

% % % % % % % % % % % % % % % % % % % % % % %

\begin{exer}[7.2.17]
    Use the class equation to show that a group of order $pq$, with $p$
    and $q$ prime, contains an element of order $p$.
\end{exer}

\begin{proof}

\end{proof}

% % % % % % % % % % % % % % % % % % % % % % % % % % % % % % % % % %
% % % % % % % % % % % % % % % % % % % % % % % % % % % % % % % % % %
\section*{Pre-Lecture Problems}

\begin{exer}[6.7.1]
    Let $G = D_4$ be the dihedral group of symmetries of the square.
    \begin{enumerate}
        \item What is the stabilizer of a vertex? Of an edge?
        \item $G$ operates on the set of two elements consisting of the
            diagonal lines. What is the stabilizer of a diagonal?
    \end{enumerate}
\end{exer}

\begin{proof}

\end{proof}

% % % % % % % % % % % % % % % % % % % % % % % % % % % % % % % % % %

\begin{exer}[6.8.2]
    What is the stabilizer of the coset $[aH]$ for the operation of $G$
    on $G/H$?
\end{exer}

\begin{proof}

\end{proof}

% % % % % % % % % % % % % % % % % % % % % % % % % % % % % % % % % %

\begin{exer}[7.1.1]
    Does the rule $g * x = x g^{-1}$ define an operation of $G$ on $G$?
\end{exer}

\begin{proof}

\end{proof}

% % % % % % % % % % % % % % % % % % % % % % % % % % % % % % % % % %

\begin{exer}[7.2.2]
    A group of order 21 contains the conjugacy class $C(x)$ of order 3.
    What is the order of $x$ in the group?
\end{exer}

\begin{proof}

\end{proof}

\end{document}
