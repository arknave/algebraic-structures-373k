\documentclass[12pt]{article}
\usepackage{amsfonts, amsmath, amsthm}

\setlength{\parskip}{1ex}
\setlength{\parindent}{0pt}

\newtheorem*{exer}{Exercise}

\newcommand{\img}{\text{img }}
\newcommand{\lcm}{\text{lcm }}
\newcommand{\aut}{\text{Aut }}
\newcommand{\cycle}[1]{(\mathbf{#1})}

\begin{document}

\textbf{Homework 4 -- Algebraic Structures} \\

\hrule

\begin{minipage}{.80\linewidth}
    \flushleft
    Ch 7: 3.1, 3.3, 3.4, 4.7, 4.9, 5.3, 5.6, 5,7, 5.12, 6.4, 6.5 \\
    Pre-lect: 5.1 \\
\end{minipage}
\begin{minipage}{.20\linewidth}
    \flushright
    Blake Griffith
\end{minipage}

\begin{exer}[7.3.1]

    Prove the Fixed Point Theorem (7.3.2).

\end{exer}

\begin{proof}

\end{proof}

% % % % % % % % % % % % % % % % % % % % % % %

\begin{exer}[7.3.3]

    A nonabelian group $G$ has order $p^3$, where $p$ is prime.

    \begin{enumerate}
        \item What are the possible orders of the center $Z$?

        \item Let $x$ be an element of $G$ that isn't in $Z$. What is
            the order of its centralizer $Z(x)$?

        \item What are the possible class equations for $G$?
    \end{enumerate}

\end{exer}

\begin{proof}

\end{proof}

% % % % % % % % % % % % % % % % % % % % % % %

\begin{exer}[7.3.4]

    Classify groups of order 8.

\end{exer}

\begin{proof}

\end{proof}

% % % % % % % % % % % % % % % % % % % % % % %

\begin{exer}[7.4.7]

    Let $G$ be a group of order $n$ that operates nontrivially on a set
    of order $r$. Prove that if $n > r!$, then $G$ has a proper normal
    subgroup.

\end{exer}

\begin{proof}

\end{proof}

% % % % % % % % % % % % % % % % % % % % % % % % % % % % % % % % % %

\begin{exer}[7.4.9]

    Let $x$ be an element of a group $G$, not the identity, whose
    centralizer $Z(x)$ has order $pq$, where $p$ and $q$ are primes.
    Prove that $Z(x)$ is abelian.

\end{exer}

\begin{proof}

\end{proof}

% % % % % % % % % % % % % % % % % % % % % % % % % % % % % % % % % %

\begin{exer}[7.5.3]

    Determine the orders of the elements of the symmetric group $S_7$.

\end{exer}

\begin{proof}

\end{proof}

% % % % % % % % % % % % % % % % % % % % % % % % % % % % % % % % % %

\begin{exer}[7.5.6]

    Find all subgroups of $S_4$ of order 4, and decide which ones are
    normal.

\end{exer}

\begin{proof}

\end{proof}

% % % % % % % % % % % % % % % % % % % % % % % % % % % % % % % % % %

\begin{exer}[7.5.7]

    Prove that $A_n$ is the only subgroup of $S_n$ of index 2.

\end{exer}

\begin{proof}

\end{proof}

% % % % % % % % % % % % % % % % % % % % % % % % % % % % % % % % % %

\begin{exer}[7.5.12]

    Determine the class equations of $S_6$ and $A_6$.

\end{exer}

\begin{proof}

\end{proof}

% % % % % % % % % % % % % % % % % % % % % % % % % % % % % % % % % %

\begin{exer}[7.6.4]

    Let $H$ be a normal subgroup of prime order $p$ in a finite group
    $G$. Suppose that $p$ is the smallest prime that divides the order
    of $G$. Prove that $H$ is in the center $Z(G)$.

\end{exer}

\begin{proof}

\end{proof}

% % % % % % % % % % % % % % % % % % % % % % % % % % % % % % % % % %

\begin{exer}[7.6.5]

    Let $p$ be a prime integer and let $G$ be a $p$-group. Let $H$ be a
    proper subgroup of $G$. Prove that the normalizer $N(H)$ of $H$ is
    strictly larger than $H$, and that $H$ is contained in a normal
    subgroup of index $p$.

\end{exer}

\begin{proof}

\end{proof}

% % % % % % % % % % % % % % % % % % % % % % % % % % % % % % % % % %
% % % % % % % % % % % % % % % % % % % % % % % % % % % % % % % % % %

\section*{Pre-Lecture Problems}

\begin{exer}[7.5.1]

    \begin{enumerate}
        \item Prove that the transpositions $\cycle{12}, \cycle{23},
            \dots \cycle{n - 1, n}$ generate the symmetric group $S_n$. 

        \item How many transpositions are needed to write the cycle
            $\cycle{123 \dots n}$?

        \item Prove that the cycles $\cycle{12 \dots n}$ and
            $\cycle{12}$ generate the symmetric group $S_n$.
    \end{enumerate}

\end{exer}

\begin{proof}

\end{proof}

% % % % % % % % % % % % % % % % % % % % % % % % % % % % % % % % % %

\end{document}
